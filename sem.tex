% handout
\documentclass[10pt,final,usepdftitle=false]{beamer}
\mode<presentation>
\usetheme{default}
\usepackage[french]{babel}
\usepackage[utf8]{inputenc}

\hypersetup{pdftitle={Sciences en Marche}}
\title[Sciences en Marche]{Sciences en Marche\\[0.5em]Pourquoi râle-t-on ?}
\author[Lucas Nussbaum]{Lucas Nussbaum\\[0.5em] \small  \texttt{lucas.nussbaum@univ-lorraine.fr}\\[0.5em]
source : \url{https://github.com/lnussbaum/slides-sem}}
\date{}
\usepackage{eurosym}

\setbeamersize{text margin left=1em,text margin right=1em}

\begin{document}

\frame{\titlepage}

\begin{frame}{Mon métier: enseignant-chercheur}
\begin{itemize}
\item 50\% enseignant, 50\% chercheur
\end{itemize}
\end{frame}

\begin{frame}{Mon métier: \underline{enseignant}-chercheur}
\begin{itemize}
\item Donner des cours, 192 heures (de TD) par an
\item Préparer et mettre à jour les supports (veille nécessaire)
\item Préparer les examens, évaluer les étudiants
\item Participer aux jurys
\item Gérer les problèmes divers des étudiants
\item Coordonner les interventions des intervenants professionnels
\item Participer à l'évolution des formations (réunions pédagogiques, dossiers d'habilitation, discussions sur le contenu des modules, etc.)
\item Participer à l'administration des composantes d'enseignement
\end{itemize}
\end{frame}

\begin{frame}{Mon métier: enseignant-\underline{chercheur}}
\begin{itemize}
\item Faire de la recherche:
	\begin{itemize}
		\item Soi-même
		\item En encadrant des doctorants, ingénieurs, stagiaires
	\end{itemize}
	\smallskip
\item Écrire des publications scientifiques, les présenter en conférence
\item Écrire des propositions de projets nationaux (ANR) ou européens pour recevoir du financement
\item Participer à la gestion des grandes infrastructures de recherche (plates-formes expérimentales)
\item Participer à l'administration de la recherche (équipe de recherche, département de recherche, laboratoire)
\item Participer à l'évaluation des projets et travaux des autres
\end{itemize}
\end{frame}

\begin{frame}{Mon métier: enseignant-chercheur}
\begin{itemize}
\item Dans l'ensemble, un métier:
\begin{itemize}
\item Intéressant, varié, à la pointe de la connaissance et de la technologie
\item Où on a l'impression d'être utile
\item Avec beaucoup de liberté
\item Plutôt bien payé (même si, après 9 ans et un doctorat, je gagne toujours moins que les jeunes diplômés de mon école d'ingénieur)
\end{itemize}
\pause
\item Mais ce métier change \ldots
\end{itemize}
\end{frame}

\begin{frame}{Difficultés budgétaires des universités}
\begin{itemize}
\item 2007: Loi relative aux libertés et responsabilités des universités (LRU)
\item Transfert de \textsl{Responsabilités et Compétences Élargies} (RCE) de l'État aux universités
\item Notamment la gestion de la masse salariale
\item Problème: Glissement Vieillesse Technicité (GVT) mal pris en compte:
	dans beaucoup d'universités, la masse salariale augmente mécaniquement en vieillissant
\item Les universités sont étranglées, et choisissent de supprimer des postes (\textsl{geler}, car 
    suppression temporaire en théorie)
    \begin{itemize}
	\item Environ 50\% des postes à pourvoir l'année dernière à l'UL
	\end{itemize}
\end{itemize}
\end{frame}

% \begin{frame}{Recherche financée par projets}
% \begin{itemize}
% \item De moins en moins de financements \textsl{automatiques} (= \textsl{dotation})
% \smallskip
% \item C'est aux chercheurs de trouver des financements
% \begin{itemize}
% 	\item Agence Nationale de la Recherche (ANR):
% 	\begin{itemize}
% 		\item Budget lui aussi en baisse, moins de projets financés
% 		\item  8,75\% de taux de réussite en 2014\\
% 			$\leadsto$ beaucoup de bons projets rejetés\\
% 			$\leadsto$ énorme perte de temps
% 	\end{itemize}
% 	\smallskip
% \item Pas beaucoup mieux au niveau européen
% \end{itemize}
% \smallskip
% \item Nous avons de moins en moins les moyens de travailler
% \end{itemize}
% \end{frame}

\begin{frame}{Difficultés budgétaires: conséquences}
\begin{itemize}
\item Il faut répartir la charge de travail des non-remplacés sur leurs collègues
\item Heures supplémentaires (\textsl{complémentaires}) subies
	\begin{itemize}
	\item Casse l'objectif de 50\% enseignement / 50\% recherche
	\end{itemize}
\item Intervenants extérieurs supplémentaires à trouver\\
	($\approx$25\% des heures au département Informatique)
\item De moins en moins de postes permanents:
\begin{itemize}
\item De plus en plus de CDD (précarité)
\item Pour des besoins permanents (ex: personnel de support/soutien):\\ re-former sans arrêt
%\smallskip
%\item Moins de perspectives en France pour les doctorants
\end{itemize}
\end{itemize}
\end{frame}

\begin{frame}{Non-solutions}
\begin{itemize}
\item Supprimer des groupes d'étudiants
\item Fermer des formations
\item Sélectionner à l'entrée pour plus de formations
\item Ajouter des \textsl{numerus clausus}
\item Augmenter les frais d'inscription à l'université
\item Contribution exceptionnelle des étudiants
\item Diminuer la qualité de l'enseignement\\ {\small (groupes plus nombreux, un enseignant pour deux groupes de TP, etc.)}
\end{itemize}
\end{frame}

\begin{frame}{Vraies solutions ?}
\begin{itemize}
\item Considérer qu'avoir un système d'enseignement supérieur et de recherche de 
qualité est important, surtout en période de crise ?
\pause
\item Rediriger une partie du financement du Crédit Impôt Recherche ?
\begin{itemize}
\item Mécanisme permettant aux entreprises de bénéficier d'un crédit d'impôt sur leurs dépenses de R\&D
\item Environ 6 Md\euro/an, soit l'équivalent du financement de l'ensemble des 
organismes de recherche (Cnrs, Inserm, CEA, Inra, Inria, IRD, Ifremer...)
\item Inefficace d'après la Cour des Comptes:\\
	{\sl \og L’évolution qu’a connue la dépense intérieure de R\&D des entreprises n’est pas à ce jour en proportion de l’avantage fiscal accordé aux entreprises. \fg }
\item Trop peu contrôlé
\item Profite surtout aux grands groupes (optimisation fiscale)
\end{itemize}
\end{itemize}
\end{frame}

\end{document}
